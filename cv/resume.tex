\documentclass[12pt, a4paper]{awesome-cv}
\fontdir[fonts/]
\newcommand*{\sectiondir}{resume/}
%%% Override color
% Awesome Colors: awesome-emerald, awesome-skyblue, awesome-red, awesome-pink, awesome-orange, awesome-nephritis, awesome-concrete, awesome-darknight
\colorlet{awesome}{black}
%\definecolor{awesome}{HTML}{CA63A8}
%% Colors for text
%\definecolor{darktext}{HTML}{414141}
%\definecolor{text}{HTML}{414141}
%\definecolor{graytext}{HTML}{414141}
%\definecolor{lighttext}{HTML}{414141}
%%% Override a separator for social informations in header(default: ' | ')
%\headersocialsep[\quad\textbar\quad]
\usepackage{import}
\usepackage{enumitem}
\usepackage{hyperref}
\hypersetup{
    colorlinks=true,
    pdfborder={0 0 0},
    allcolors=genetics,
}

% Contact Info
\name{Andrew A.}{Zaffos}
\address{1955 East Sixth Street, PO Box 210184 Tucson, AZ 85721}
\mobile{1-(520)-621-3635} 
%%% Social
\email{azaffos@email.arizona.edu}
\homepage{www.azstrata.org}
\github{aazaff}
%\linkedin{posquit0}
%%% Optionals
\position{Data Scientist{\enskip\cdotp\enskip}Geologist{\enskip\cdotp\enskip}Paleobiologist{\enskip\cdotp\enskip}Developer}
%\quote{``Make the change that you want to see in the world."}

%%% Make a footer for CV with three arguments(<left>, <center>, <right>)
\makecvfooter
  {\today}
  {Andrew A. Zaffos~~~·~~~Curriculum Vitae}
  {\thepage}

\begin{document}
%%% Make a header for CV with personal data
\makecvheader

%%% Import contents
\import{\sectiondir}{presentation.tex}

\import{\sectiondir}{education.tex}
\vspace{-20pt}

\cvsection{Published Papers}
\fontsize{10pt}{1em}\bodyfontlight\upshape\color{text}
\begin{itemize}[leftmargin=*]
\item{Holland, S.M. and \textbf{A. Zaffos} (2011). Niche conservatism along an onshore-offshore gradient. \textit{Paleobiology} 37:270-286.}
\item{\textbf{Zaffos, A.} and S.M. Holland (2012). Abundance and extinction in Ordovician-Silurian brachiopods, Cincinnati Arch, Kentucky and Ohio. \textit{Paleobiology} 38:278-291.}
\item{\textbf{Zaffos, A.} and A.I. Miller (2015). Cenozoic latitudinal response curves: Individualistic changes in the latitudinal distributions of marine bivalves and gastropods. \textit{Paleobiology} 41:33-44.}
\item{Brett, C.E., \textbf{Zaffos, A.}, and A.I. Miller (2016). Niche conservatism, tracking, and ecological stasis: A hierarchichal perspective. Eldredge, N. and T. Pievani (Eds.), Evolutionary Theory: A Hierarchical Perspective, University of Chicago Press.}
\item{\textbf{Zaffos, A.}, Finnegan, S., and S.E. Peters (\textit{2017}). Plate-tectonic regulation of marine animal diversity. \textit{Proceedings of the National Academy of Sciences} 114:5653–5658.}
\item{Peters, S.E., Ross, I., Czaplewski, J., Glassel, A., Husson, J., Syverson, V., \textbf{Zaffos, A.}, and M. Livny (2017). A new tool for deep-down data mining. \textit{Eos} 98.}
\end{itemize}

\cvsection{Published Abstracts}
\fontsize{10pt}{1em}\bodyfontlight\upshape\color{text}
\begin{itemize}[leftmargin=*]
\item{\textbf{Zaffos, A.}, Bonnette, M., Christie, M., Lunze, J., Pryor, A.L., and K.M. Layou (2008). Implications of beta: Diversity partitioning during Miocene-Pliocene background extinction. \textit{Geological Society of America Abstracts with Programs} 40:60.}
\item{Holland, S.M. and \textbf{A. Zaffos} (2009). Testing the importance of incumbency: The conservation of ecological response curves along an onshore-offshore gradient. \textit{9th North American Paleontological Convention Abstracts, Cincinnati Museum Center Scientific Contributions} 3:145.}
\item{\textbf{Zaffos, A.} and S.M. Holland (2010). Brachiopod abundance and extinction in the Late Ordovician and Early Silurian: Background effects during a mass extinction. \textit{Geological Society of America Abstracts with Programs} 42:481.}
\item{\textbf{Zaffos, A.} (2011). Latitudinal preferences in Cenozoic mollusks: A change from conservatism to dynamism with the passage of time. \textit{Geological Society of America Abstracts with Programs} 43:541.}
\item{\textbf{Zaffos, A.}, Brett, C.E., and A.I. Miller (2014). The persistence of ecological gradients: What do we really know? \textit{10th North American Paleontological Convention Abstracts, The Paleontological Society Special Publication} 13:164-165.}
\item{\textbf{Zaffos, A.}, Miller, A.I., and C.E. Brett (2014). A numerical model of gradient change and its implications for empirical data. \textit{Geological Society of America Abstracts with Programs} 46:136.}
\item{Peters, S., Husson, J., Czaplewski, J., and \textbf{A. Zaffos} (2015). Macrostrat: Towards a common data infrastructure to support deep time earth systems science. \textit{Geological Society of America Abstracts with Programs} 47:70.}
\item{\textbf{Zaffos, A.}, Peters, S.E., Husson, J., and J. Czaplewski (2015). A comparative estimate of different biodiversity curves and comments on the link between rock availability and biodiversity. \textit{Geological Society of America Abstracts with Programs} 47:71.}
\item{\textbf{Zaffos, A.} Peters, S.E., and S.K. McMullen (2015). Empirical patterns of geographic range size expansion and contraction in marine invertebrates and terrestrial mammals are well predicted by a random walk. \textit{Geological Society of America Abstracts with Programs} 48:668.}
\item{\textbf{Zaffos, A.}, Peters, S.E., and S. Finnegan (2016). Plate-tectonic regulation of biodiversity and continental endemism. \textit{Geological Society of America Abstracts with Programs} 48:152.}
\item{Hartman, S. Lovelace, D.M., Linzmeier, B.J., Fitch, A., Kufner, A., \textbf{Zaffos, A.}, Matthewson, P., and W. Porter (2016) Mechanistic physiological modelling and geographic distribution of Late Triassic tetrapods. \textit{Society of Vertebrate Paleontology Annual Meeting}.}
\item{Ito, E.T., \textbf{Zaffos, A.}, Syverson, V.J., Ross, I.A., and S.E. Peters (2017). Identifying, cross-referencing, and extracting dark data using GeoDeepDive. \textit{Digital Data in Biodiversity Research Conference}.}
\item{\textbf{Zaffos, A.} and S.E Peters (2017). Reassessing our expectations for marine latitudinal biodiversity gradients in modern and ancient systems. \textit{Geological Society of America Abstracts with Programs} 49:6.}
\item{Peters, S.E., Syverson, V.J., \textbf{Zaffos, A.}, Husson, J., Ross, I. and J. Czaplewski (2017). Extending the reach and resolution of the paleobiology database with computational and data infrastructures. \textit{Geological Society of America Abstracts with Programs} 49:6.}
\end{itemize}

\cvsection{Invited Talks, Chapters, and Lectures}
\fontsize{10pt}{1em}\bodyfontlight\upshape\color{text}
\begin{itemize}[leftmargin=*]
\item{\textbf{Zaffos, A.} (2012). Climate Change: A paleontologist's perspective. \textit{Illinois Wesleyan University}.}
\item{\textbf{Zaffos, A.} (2012). Mass Extinction: What you need to know so it doesn't happen to you. \textit{Northern Kentucky University}.}
\item{Brett, C.E., \textbf{Zaffos, A.}, Baird, G.C., and A.J. Bartholomew (2013). Fossil beds, facies gradients, and seafloor dynamics in the Middle Devonian Moscow Formation, Western New York. Deakin, A.K. and G.G. Lash (Eds.), \textit{New York State Geological Association 85th Annual Field Trip Guidebook}, SUNY Fredonia.}
\item{\textbf{Zaffos, A.} (2014). Looking for first principles in paleobiology. \textit{Centre College}.}
\item{\textbf{Zaffos, A.} (2015). Reconsidering the Age-Area hypothesis with R and Big Data. \textit{Madison R Programming Group}.}
\item{\textbf{Zaffos, A.} (2016). The dynamics of geographic range size. \textit{Madison Evolution Seminar Series}.}
\item{\textbf{Zaffos, A.} (2017). Geobiology: The history of early life. \textit{University of Wisconsin-Madison}.}
\item{\textbf{Zaffos, A.} (2017). Global tectonics regulate marine biodiversity. \textit{NOAA Science Seminar Series}.}
\item{\textbf{Zaffos, A.} (2017). Global tectonics regulate marine biodiversity. \textit{Arizona Geological Society}.}
\item{\textbf{Zaffos, A.} (2017). Rethinking our expectations for global biogeographic patterns and processes. \textit{University of Arizona}.}
\item{\textbf{Zaffos, A.} and L. Bookman (2017). A flexible framework for data preservation and distribution. \textit{NGGDPP Data Preservation and Rescue Workshop.}}
\end{itemize}

\import{\sectiondir}{extracurricular.tex}
\vspace{-20pt}
\import{\sectiondir}{experience.tex}
\vspace{-20pt}

\cvsection{Scholarships and Research Support}
\fontsize{10pt}{1em}\bodyfontlight\upshape\color{text}
\begin{itemize}[leftmargin=*]
\item{Hawthorne Rotary International (2004)}
\item{University of Georgia Graduate Assistantship (2008)}
\item{Geological Society of America Graduate Grant (2009)}
\item{Paleontological Society, Cooper Grant (2009)}
\item{University of Georgia Miriam Watts-Wheeler Memorial Grant (2009)}
\item{University of Cincinnati Klekamp Grant (2011)}
\item{University of Cincinnati Kenneth E. Caster Memorial Grant (2011)}
\item{Paleontological Society of America, Gould Grant (2011)}
\item{Geological Society of America Graduate Grant (2012)}
\item{American Museum of Natural History: Theodore Roosevelt Grant (2012)}
\item{University of Cincinnati Wycoff Fellowship (2013)}
\item{University of Cincinnati Kenneth E. Caster Memorial Fellowship (2014)}
\item{University of Cincinnati George and Frances Rawlinson Fellowship (2014)}
\end{itemize}

\cvsection{Laboratories and Development Teams}
\fontsize{10pt}{1em}\bodyfontlight\upshape\color{text}
\begin{itemize}[leftmargin=*]
\item{\href{http://strata.uga.edu/}{University of Georgia Stratigraphy Lab} (2008-2010)}
\item{\href{https://macrostrat.org/}{Macrostrat Development Team} (2015-Present)}
\item{\href{https://geodeepdive.org/}{GeoDeepDive Development Team} (2015-Present)}
\item{\href{https://cran.r-project.org/web/packages/velociraptr/index.html}{velociraptr R Package Developer} (2015-Present)}
\item{\href{https://steppe.org/epandda/}{ePANDDA Working Group} (2016-Present)}
\item{\href{https://paleobiodb.org/#/people}{Paleobiology Database Developer} (2016-Present)}
\item{\href{http://fc.umn.edu/}{Flyover Country Working Group} (2017-Present)}
\end{itemize}

\cvsection{Community Participation}
\fontsize{10pt}{1em}\bodyfontlight\upshape\color{text}
\begin{itemize}[leftmargin=*]
\item{Alpha Phi Omega Service Organization (2005-2008)}
\item{Geological Society of America (2008-Present)}
\item{Paleontological Society (2008-Present)}
\item{Osher Lifelong Learning Center, Teaching Assistant (2010)}
\item{American Society of Naturalists (2010-Present)}
\item{Boy Scouts of America, Geology Merit Badge Counselor (2011-2013)}
\item{Paleobiology Database Hackathon, University of California - Santa Cruz (2015)}
\item{\textbf{\textit{Paleobiology}} Reviewer (2015, 2017)}
\item{\textbf{\textit{Journal of Biogeography}} Reviewer (2016)}
\item{\textbf{\textit{Nature: Scientific Reports}} Reviewer (2016)}
\item{\textbf{\textit{Geology}} Reviewer (2017)}
\item{Data Refuge Madison (2017)}
\item{Paleontological Society: Career and Mentoring Panelist (2017)}
\end{itemize}

\cvsection{Current Projects}
\fontsize{10pt}{1em}\bodyfontlight\upshape\color{text}
\begin{itemize}[leftmargin=*]
\item{\textbf{Zaffos, A.}, Peters, S.E., and S.M. McMullen (\textit{in preparation}). The dynamics of geographic range size over geologic timescales.}
\item{McMullen, S.M., Peters, S.E., Weissman, G., and \textbf{A. Zaffos} (\textit{in preparation}). Quality of the terrestrial mammalian fossil record.}
\item{Ito, E.T., \textbf{Zaffos, A.}, and S.E. Peters (\textit{in preparation}). The quality of the Paleobiology Database.}
\item{\textbf{Zaffos, A.} (\textit{in preparation}). Reassessing our expectations for modern and ancient marine latitudinal biodiversity gradients.}
\end{itemize}

\import{\sectiondir}{honors.tex}

\end{document}
